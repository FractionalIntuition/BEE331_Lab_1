\documentclass[9pt,a4paper]{report}
\usepackage{mwe}
\usepackage{listings}
\usepackage{amsmath}
\usepackage{graphicx}
\usepackage{subfig}
\usepackage{float}
\usepackage{xcolor}
\usepackage{multirow}
\usepackage{hyperref}
\usepackage{fancyhdr}
\usepackage{sectsty}
\usepackage[dvipsnames]{xcolor}
\usepackage{soul}
\usepackage[compact]{titlesec}
\usepackage{float}
\usepackage[left=0.5cm,right=0.5cm,top=0.5cm,bottom=0.5cm]{geometry}
\graphicspath{{BEE331_Lab_1/}}

\newcommand*{\nchapter}[1]{%
	\chapter*{#1}%
	\addcontentsline{toc}{chapter}{#1}
	\vspace{-14mm}}
\newcommand*{\nsection}[1]{%
	\section*{#1}%
	\addcontentsline{toc}{section}{#1}}
\newcommand*{\nsubsection}[1]{%
	\subsection*{#1}%
	\addcontentsline{toc}{subsection}{#1}}
\newcommand*{\nsubsubsection}[1]{%
	\subsubsection*{#1}%
	\addcontentsline{toc}{subsubsection}{#1}}

\chaptertitlefont{\large}
\sectionfont{\normalsize}
\fontsize{9}{11}\selectfont
\begin{document}
	\textsc{Testing}
	\newline
	//Add screenshots here
	\newline
	\hspace*{.15in}
	i. The first limiter circiut reduces peak voltage showing the behavior of the exponetial diodes. This make the output wave more square than the input wave because it is limiting the input before it can start curving back down. The second limiter circuit (find picture to compare)
	\newline
	\hspace*{.15in} 
	ii.
	\newline
	\hspace*{.15in} 
	iii. We changed the coupling setting if there was noise in our signals.
	\newline
	\hspace*{.15in} 
	iv. You should be able to make the output signal the same by staying in the region where both diodes are in an "off" state. This will cause the cicuit to not be limited by current through the diodes.
	\newline
	
	\vspace*{1.5in}

\end{document}